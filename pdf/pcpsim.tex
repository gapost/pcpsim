\documentclass[12pt,a4paper]{article}

\usepackage[utf8]{inputenc}
\usepackage{amsmath}
\usepackage{amsfonts}
\usepackage{amssymb}
\usepackage{graphicx}
    
%\usepackage[left=2cm,right=2cm,top=2cm,bottom=2cm]{geometry}
    
\usepackage[round]{natbib}
\usepackage{hyperref}

%%%%%%%%%%%%%%%% Code Listing %%%%%%%%%%%%%%%%%%%%%%%%%%%%%%%
\usepackage{listings}
\usepackage{color}

\definecolor{mygreen}{rgb}{0,0.6,0}
\definecolor{mygray}{rgb}{0.5,0.5,0.5}
\definecolor{mymauve}{rgb}{0.58,0,0.82}

\lstset{ %
  backgroundcolor=\color{white},   % choose the background color; you must add \usepackage{color} or \usepackage{xcolor}
  basicstyle=\footnotesize,        % the size of the fonts that are used for the code
  breakatwhitespace=false,         % sets if automatic breaks should only happen at whitespace
  breaklines=true,                 % sets automatic line breaking
  captionpos=b,                    % sets the caption-position to bottom
  commentstyle=\color{mygreen},    % comment style
  deletekeywords={...},            % if you want to delete keywords from the given language
  escapeinside={\%*}{*)},          % if you want to add LaTeX within your code
  extendedchars=true,              % lets you use non-ASCII characters; for 8-bits encodings only, does not work with UTF-8
  frame=single,                    % adds a frame around the code
  keepspaces=true,                 % keeps spaces in text, useful for keeping indentation of code (possibly needs columns=flexible)
  keywordstyle=\color{blue},       % keyword style
  language=Matlab,                 % the language of the code
  morekeywords={*,...},            % if you want to add more keywords to the set
  numbers=left,                    % where to put the line-numbers; possible values are (none, left, right)
  numbersep=5pt,                   % how far the line-numbers are from the code
  numberstyle=\tiny\color{mygray}, % the style that is used for the line-numbers
  rulecolor=\color{black},         % if not set, the frame-color may be changed on line-breaks within not-black text (e.g. comments (green here))
  showspaces=false,                % show spaces everywhere adding particular underscores; it overrides 'showstringspaces'
  showstringspaces=false,          % underline spaces within strings only
  showtabs=false,                  % show tabs within strings adding particular underscores
  stepnumber=2,                    % the step between two line-numbers. If it's 1, each line will be numbered
  stringstyle=\color{mymauve},     % string literal style
  tabsize=2                        % sets default tabsize to 2 spaces
}
%%%%%%%%%%%%%%%%%%%%%%%%%%%%%%%%%%%%%%%%%%%%%%%%%%%%%%%

\usepackage{textcomp}
\author{
  A. Theodorou\\
  \texttt{theoda@ipta.demokritos.gr}

}
\date{June 2020}
\title{\texttt{pcpsim} \\ 
MODEL FOR HOMOGENEOUS PRECIPITATION KINETICS in \texttt{GNU OCTAVE}
}
     
\begin{document}

\maketitle

\section{Introduction}
The model for homogeneous isothermal precipitation is partly based on the model by Langer and Schwartz, modified by Kampmann and Wagner (MLS model). In this model, we treated the nucleation and growth of the precipitation.


\section{Thermodynamics of the model}

As a first step we need to define the driving force for precipitation at any given time of the aging process.
 
\begin{equation}
\Delta g = - \frac{kT}{V_{at}} \cdot S 
\end{equation}

with

\begin{equation}
S =  X_p \ln(\frac{X_C}{X_{eq}} ) - (1 - X_p) \ln(\frac{1 - X_C}{X_{eq}}) 
\end{equation}

where $V_{at}$ is the atomic volume (considered as constant for all species), $S$ is a thermodynamical  function giving the driving force for nucleation (based
on the hypothesis of a diluted and regular solid solution), $X_{eq}$ is the equilibrium solute mole fraction in the matrix, $X_p$ the nitrogen mole fraction in the precipitate, and $X_C$ the current solute mole fraction of the matrix.

The nucleation rate as the derivative of the precipitation density $N$ is:

\begin{equation}
\frac{d N}{d t} = N_0 Z \beta^* \exp(-\frac{\Delta G^*}{kT}) \exp(-\frac{\tau}{t})
\end{equation}

where $N_0$ is the number of nucleation sites per unit volume ($\approx  l/vat =2/a^3$ for a bcc structure with lattice parametera), Z is the Zeldovich factor ($\approx$ 1/20), $\tau$ is the incubation time. The other parameters of equation are expressed as follows:

\begin{subequations}
	\begin{align}
\beta^* = \frac{4\pi R^{*2} D X_{C0}}{a^4} \\
R^* = \frac{R_0}{S} \\
R_0 = \frac{2\gamma V_{at}}{kT} \\
\Delta G^* = \frac{16}{3}\pi \frac{\gamma}{\delta g} \\
\tau = \frac{1}{2\beta^* Z} \\
	\end{align}
	
\end{subequations}

where $R^*$ is the nucleation critical radius, R0 is a thermodynamical parameter
which has the dimension of a length, $\gamma$ is the matrix/precipitate interfacial energy, $D$ is the diffusion coefficient of solute atoms in the matrix, $X_{C0}$ is the initial solute mole fraction.

The precipitate size increase during a time increment dt is then calculated as:
\begin{equation}
\frac{dR}{dt} = \frac{D}{R} \cdot \frac{X_C - X_{eq} exp(R_0/(X_pR))}{X_p - X_{eq} exp(R_0/(X_pR))} + \frac{1}{N}\frac{dN}{dt} \cdot (\alpha R^* - R)
\end{equation}

The first term corresponds to the growth of existing precipitates (including the Gibbs-Thomson effect) and the second term to the appearance of the new nuclei of
size $R^*$. The numerical factor $\alpha = 1.05$ results from the fact that new precipitates only grow if their size is slightly larger than the nucleation size. 
Finally, the coupling between the precipitation density and their mean radius is made through the solute balance:

\begin{equation}
X_c = \frac{X_{C0}-(4/3)\pi (X_pNR^3)}{1-(4/3)\pi (NR^3)}
\end{equation}

Now are defined the new dimensionless variables which are more easy to use for programming.


\bibliographystyle{plainnat}
\bibliography{library}

\end{document}
 