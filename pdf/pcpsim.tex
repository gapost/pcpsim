\documentclass[12pt,a4paper]{article}

\usepackage[utf8]{inputenc}
\usepackage{amsmath}
\usepackage{amsfonts}
\usepackage{amssymb}
\usepackage{graphicx}
    
%\usepackage[left=2cm,right=2cm,top=2cm,bottom=2cm]{geometry}
    
\usepackage[round]{natbib}
\usepackage{hyperref}

%%%%%%%%%%%%%%%% Code Listing %%%%%%%%%%%%%%%%%%%%%%%%%%%%%%%
\usepackage{listings}
\usepackage{color}

\definecolor{mygreen}{rgb}{0,0.6,0}
\definecolor{mygray}{rgb}{0.5,0.5,0.5}
\definecolor{mymauve}{rgb}{0.58,0,0.82}

\lstset{ %
  backgroundcolor=\color{white},   % choose the background color; you must add \usepackage{color} or \usepackage{xcolor}
  basicstyle=\footnotesize,        % the size of the fonts that are used for the code
  breakatwhitespace=false,         % sets if automatic breaks should only happen at whitespace
  breaklines=true,                 % sets automatic line breaking
  captionpos=b,                    % sets the caption-position to bottom
  commentstyle=\color{mygreen},    % comment style
  deletekeywords={...},            % if you want to delete keywords from the given language
  escapeinside={\%*}{*)},          % if you want to add LaTeX within your code
  extendedchars=true,              % lets you use non-ASCII characters; for 8-bits encodings only, does not work with UTF-8
  frame=single,                    % adds a frame around the code
  keepspaces=true,                 % keeps spaces in text, useful for keeping indentation of code (possibly needs columns=flexible)
  keywordstyle=\color{blue},       % keyword style
  language=Matlab,                 % the language of the code
  morekeywords={*,...},            % if you want to add more keywords to the set
  numbers=left,                    % where to put the line-numbers; possible values are (none, left, right)
  numbersep=5pt,                   % how far the line-numbers are from the code
  numberstyle=\tiny\color{mygray}, % the style that is used for the line-numbers
  rulecolor=\color{black},         % if not set, the frame-color may be changed on line-breaks within not-black text (e.g. comments (green here))
  showspaces=false,                % show spaces everywhere adding particular underscores; it overrides 'showstringspaces'
  showstringspaces=false,          % underline spaces within strings only
  showtabs=false,                  % show tabs within strings adding particular underscores
  stepnumber=2,                    % the step between two line-numbers. If it's 1, each line will be numbered
  stringstyle=\color{mymauve},     % string literal style
  tabsize=2                        % sets default tabsize to 2 spaces
}
%%%%%%%%%%%%%%%%%%%%%%%%%%%%%%%%%%%%%%%%%%%%%%%%%%%%%%%

\usepackage{textcomp}
\author{
  A. Theodorou\\
  \texttt{theoda@ipta.demokritos.gr}
}
\date{June 2020}
\title{\texttt{pcpsim} \\ 
MODEL FOR HOMOGENEOUS PRECIPITATION KINETICS in \texttt{GNU OCTAVE}
}
     
\begin{document}

\maketitle

\section{Introduction}
The model for homogeneous isothermal precipitation is partly based on the model by Langer and Schwartz, as modified by Kampmann and Wagner (MLS model) and describes the nucleation and growth of  precipitates from a solid solution.


\section{Nucleation and Growth}

As a first step we need to define the driving force for precipitation at any given time of the aging process: 
\begin{equation}
\Delta g = - \frac{kT}{V_{at}} \cdot S 
\end{equation}
with
\begin{equation}
S =  X_p \ln\frac{X_C}{X_{eq}} + (1 - X_p) \ln\frac{1 - X_C}{1-X_{eq}} 
\end{equation}
where $V_{at}$ is the atomic volume (considered as constant for all species, $V_{at}=a^3/2$ for a bcc structure with lattice parameter $a$), $S$ is a thermodynamical function giving the driving force for nucleation (based
on the hypothesis of a diluted and regular solid solution), $X_{eq}$ is the equilibrium solute mole fraction in the matrix, $X_p$ the solute mole fraction in the precipitate, and $X_C$ the current solute mole fraction of the matrix.

The nucleation rate as the time derivative of the precipitate volume density $N$ is:
\begin{equation}
\label{P_density}
\frac{d N}{d t} = N_0 Z \beta^* \exp(-\frac{\Delta G^*}{kT}) \exp(-\frac{t_i}{t})
\end{equation}
where $N_0$ is the number of nucleation sites per unit volume (for homogeneous nucleation $N_0 = 1/ V_{at}$), $Z$ is the Zeldovich factor ($\approx$ 1/20), $t_i$ is the incubation time. The other parameters of equation are expressed as follows:

\begin{subequations}
	\begin{align}
\beta^* &= \frac{4\pi R^{*2} D X_{0}}{a^4} \\
R^* &= \frac{2\gamma V_{at}}{S\,kT} \\
\Delta G^* &= \frac{4}{3}\pi R^{*2}\gamma \\
t_i &= \frac{1}{2\beta^* Z} 
	\end{align}	
\end{subequations}
where $R^*$ is the critical nucleation radius, $R_0$ is a thermodynamical parameter
which has the dimension of a length, $\gamma$ is the matrix/precipitate interfacial energy, $D$ is the diffusion coefficient of solute atoms in the matrix, $X_{0}$ is the initial solute mole fraction.

The precipitate size increase during a time increment $dt$ is then calculated as:
\begin{equation}
\label{P_radius}
\frac{dR}{dt} = \frac{D}{R} \cdot \frac{X_C - \hat{X}_{eq}}{\hat{X}_p - \hat{X}_{eq}} + \frac{1}{N}\frac{dN}{dt} \cdot (\alpha R^* - R)
\end{equation}

The first term on the left hand side of \eqref{P_radius} corresponds to the growth of existing precipitates (including the Gibbs-Thomson effect). $\hat{X}_{eq,p}$ represent the solute concentration at the interface ($r=R$) in the precipitate and the matrix, respectively, as modified by surface tension according to the Gibbs-Thomson effect. For an ideal solution they are given by \citep{Wagner-2005-HomogeneousSecond-P,Calderon-1994-Ostwaldripeningin}:
\begin{equation}
\hat{X}_{eq,p} =  X_{eq,p} \cdot \exp \left( \frac{2\gamma V_{at}}{kT\, R} \frac{1-X_{eq,p}}{X_p - X_{eq}}\right) 
\end{equation}

The second term of \eqref{P_radius} is due to the appearance of new nuclei of
size $R^*$. The numerical factor $\alpha = 1.05$ results from the fact that new precipitates only grow if their size is slightly larger than the nucleation size. 

Finally, the coupling between the precipitation density and their mean radius is made through the solute balance:
\begin{equation}
X_0 = X_C\,(1-\Phi) + X_p\,\Phi
\end{equation}
where $\Phi=\frac{4}{3}\pi R^3 N$ is the precipitate volume fraction.

\subsection{Dimensionless formulation}
Now the following dimensionless variables are defined that are easier to use for programming:
\begin{subequations}
	\begin{align}
\tau &= \frac{D\cdot t}{a^2} \\
\nu &= N / N_0 \\
R_0 &= \frac{2\gamma V_{at}}{kT} \\
\rho &= R/R_0
	\end{align}
\end{subequations}

Thus the equations (\ref{P_density}), (\ref{P_radius}) are replaced by the following dimensionless differential equations:
\begin{align}
\frac{d\nu}{d\tau} &= 
\frac{\beta_0}{S^2} 
\exp\left( -\frac{\Delta G_0}{S^2}\right)  
\exp\left( -\frac{S^2}{2\beta_0 \tau}\right)  \\
%%%%
\frac{d\rho}{d\tau} &=  
\frac{a^2}{R_0^2} 
\frac{X_C - \hat{X}_{eq}}{\hat{X}_p - \hat{X}_{eq}} 
\; \frac{1}{\rho}
+ 
\frac{1}{\nu}\frac{d\nu}{d\tau}
\left( \frac{\alpha R_0}{S} - \rho \right)
\end{align}
where
\begin{align}
\beta_0 &= 4\pi R_0^2 X_{0}Z / a^2 \\
\Delta G_0 &= \frac{4\pi R_0^2 \gamma }{3kT}  \\
\hat{X}_{eq,p} &= X_{eq,p} \exp 
\left( \frac{1}{\rho} \frac{1-X_{eq,p}}{X_p - X_{eq}}\right)
\end{align}

Differentiating the solute balance equation we obtain the change rate of $X_C$:
\begin{equation}
\frac{dX_C}{d\tau} = -(X_p - X_C)\, \frac{\Phi}{1-\Phi} \,
\left[ 3\frac{\dot{\rho}}{\rho} + \frac{\dot{\nu}}{\nu} \right]
\end{equation}
where $\Phi = \Phi_0 \rho^3 \nu$, $\Phi_0 = \frac{4}{3}\pi R_0^3 N_0$.


\section{Coarsening}
In late stages of precipitation the nucleation rate progressively decreases because the current solute concentration in the matrix ($X_C$) decreases and $X_C - X_{eq} \approx 0$, this means that it is no longer supersaturated. 

The precipitates continue to grow according to eq. \eqref{P_radius}. In the long time limit the LSW theory predicts that the volume of the precipitates grows linearly with time, $R^3 \propto t$.

The growth rate of $R$ for an ideal solution in the long time limit becomes \citep{Calderon-1994-Ostwaldripeningin}
\begin{equation}
\label{R_coarsening}
\frac{dR}{dt} = \frac{4}{27}
\;\frac{X_{eq}\, (1-X_{eq})}{(X_{p} - X_{eq} )^2}
\;\frac{D R_0}{R^2}
\end{equation}

Furthermore, $R$ becomes equal to $R^*=R_0/S$ \citep{Deschamps-1998-Influenceofpredefo} (???). In this limit the value of $S$ is
\begin{equation}
S_\infty \approx (X_C-X_{eq}) \frac{X_p-X_{eq}}{X_{eq}(1-X_{eq})}
\end{equation}
and thus
\begin{equation}
X_C - X_{eq} = \frac{X_{eq}(1-X_{eq})}{X_p-X_{eq}} \frac{R_0}{R} \propto t^{-1/3}
\end{equation}

The precipitate volume fraction tends to the constant limiting value
\begin{equation}
\Phi = \frac{X_0 - X_{C}}{X_p-X_C} \approx \frac{X_0 - X_{eq}}{X_p-X_{eq}} = \Phi_\infty
\end{equation}
and
\begin{equation}
N = \frac{3\Phi_\infty}{4\pi R^3} \propto t^{-1}
\end{equation}


\subsection{Growth and coarsening}
In late stages of precipitation the nucleation rate progressively decreases because the current solute concentration in the matrix ($X_C$) decreases and $X_C - X_{eq} \approx 0$, this means that it is no longer supersaturated. The alloy experiences a combination of growth and coarsening. When the mean radius of precipitates ($R$) is much larger than the critical radius ($R^*$) takes places the growth phase which is described by the equations:

\begin{subequations}
		\begin{align}
 \frac{d\rho}{d\tau} = \frac{a^2}{R_0^2 \rho}\frac{X_C -X_{eq}\exp(1/(X_p\rho))}{X_p - X_{eq}exp(1/(X_p\rho))}  \\
 \frac{d\nu}{d\tau} = 0 
 	\end{align} 
\end{subequations}

When the mean radius and the critical radius are equal the equations follow the LSW law :

\begin{subequations}
	\begin{align}
	\frac{d\rho}{d\tau} = \frac{4}{27}\frac{X_{eq}}{1 (or~X_p) - X_{eq}} \frac{a^2}{R_0^2 \rho} \\
	\frac{d\nu}{d\tau} = 0 
	\end{align} 
\end{subequations}

Differentiating the second equation, one can deduce the variation of the matrix solute concentration with time in pure coarsening:

\begin{subequations}
	\begin{align}
	\frac{dS}{d\tau} = \frac{\dot{X_C}(X_p - X_C)}{X_C(1-X_C)} \\
	\frac{dX_C}{d\tau} = \frac{X_C(1-X_C)}{X_C-X_P} \frac{1}{\rho^2} \frac{d\rho}{d\tau}
	\end{align} 
\end{subequations}

The rate of variation of the density of precipitates in pure coarsening is:

\begin{equation}
\frac{d\nu}{d\tau} = \frac{\dot{\rho}}{\rho} \left[  \frac{X_C-(1-X_C)}{(X_C - X_p)^2 \rho} \left( \frac{1}{h\rho^3} - \nu\right)  - 3\nu \right] 
\end{equation}

In order to go continuously from the growth stage to the coarsening stage, we de®ne a coarsening fraction:

\begin{subequations}
	\begin{align}
	\frac{d\rho}{\tau} = (1-f_{coars}) \frac{d\rho}{d\tau}~ _{growth}  + f_{coars} \frac{d\rho}{d\tau}~ _{coars}
	\frac{d\nu}{\tau} = f_{coars} \frac{d\nu}{d\tau}~ _{growth}
	\end{align} 
\end{subequations}

The $f_{coars} = 0$ when $R >> R^*$ and $f_{coars} = 1$ when $R = R^*$. The $f_{coars}$ function has the following expression which fulfills these
requirements:

\begin{equation}
f_{coars} = 1 - erf \left(4(\rho \cdot S - 1) \right)
\end{equation}

\bibliographystyle{plainnat}
\bibliography{pcpsim}

\end{document}
 