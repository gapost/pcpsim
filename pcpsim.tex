\documentclass[12pt,a4paper]{article}
\usepackage[utf8]{inputenc}
\usepackage{amsmath}
\usepackage{amsfonts}
\usepackage{amssymb}
\usepackage{graphicx}
\usepackage[left=2cm,right=2cm,top=2cm,bottom=2cm]{geometry}
\usepackage[square,numbers]{natbib}
\bibliographystyle{unsrtnat}
%\bibliographystyle{abbrvnat}
\usepackage{hyperref}
\author{AT, GA}
\title{Nucleation, growth, coarsening equations}
\begin{document}
\maketitle

\section{Intro}

The formulas in this note have been collected from a number of papers \& books:
\begin{itemize}
\item \citet{Wagner-2005-HomogeneousSecond-P}, a comprehensive review of theory and experiments
\item \citet{Russell-1980-Nucleationinsolids}, good explanation of nucleation
\item \citet{Calderon-1994-Ostwaldripeningin}, growth and coarsening in concentrated alloys with second phase precipitate concentration $\neq 1$
\item \citet{Deschamps-1998-Influenceofpredefo}, example application of theory to Al-Mg-Si
\item \citet{Perez-2003-ID509}, example application to Fe-C
\end{itemize}



\section{Definitions}

\begin{description}
\item[$\alpha$] Solid solution phase
\item[$\beta$] Precipitate phase
\item[$C_0$] Average atomic solute concentration
\item[$C_e^{\alpha,\beta}$] Equilibrium concentration in $\alpha$ or $\beta$ 
\item[$C_\infty^{\alpha}$] Concentration in the $\alpha$ matrix (far from the precipitate)
\item[$\hat{C}^{\alpha,\beta}$] Concentration in $\alpha$ or $\beta$ at the precipitate/matrix interface (modified by surface tension)
\item[$N$] Number of precipitates per unit volume
\item[$R$] Precipitate radius
\item[$\gamma$] Surface energy (J/m$^2$)
\item[$V_0$] Atomic volume
\item[$a$] Lattice constant
\item[$D$] Solute diffusion coefficient
\item[$Z$] Zeldovitch factor, typically $Z\approx 1/20$
\end{description}

\section{Model Parameters}

The value of $\gamma$ is mostly used as a model parameter to fit experimental data. From $\gamma$ the following central quantities can be obtained:

\begin{equation}
R_0 = \frac{2\gamma V_0}{k_B\,T}
\end{equation}

\begin{equation}
\Delta G_0 = \frac{4}{3}\pi R_0^2 \gamma
\end{equation}

In one case \cite{Deschamps-1998-Influenceofpredefo} both the values of $\gamma$ and $\Delta G_0$ were used as fitting parameters. However the authors did not specify exactly how they did it. We assume that $R_0$ is calculated from $\gamma$ and $\Delta G_0$ is used in the calculation of free energy, $\Delta G^*$.

\section{Nucleation}

The ``driving force for nucleation'' is proportional to 
\begin{equation}
S = C_e^\beta \log \frac{C_\infty^\alpha}{C_e^\alpha} + (1-C_e^\beta) \log \frac{1-C_\infty^\alpha}{1-C_e^\alpha}
\end{equation}

The critical nucleation radius is 
\begin{equation}
R^* = R_0 / S
\end{equation}
and the associated free energy barrier
\begin{equation}
\Delta G^* = \Delta G_0 / S^2
\end{equation}

The nucleation rate is
\begin{equation}
\frac{dN}{dt} = \frac{1}{V_0} Z \beta^* 
\exp \left( -\frac{\Delta G^*}{k_B T}\right) 
\exp \left( -\frac{1}{2 Z \beta^* t}\right)
\end{equation}
where
\begin{equation}
\beta^* =\frac{4\pi R^{*2} D C_0}{a^4}
\end{equation}


\section{Growth}

By solving the steady-state diffusion equation $\nabla^2 C(r)=0$ in the region around the precipitate the following equation is obtained
\begin{equation}
\frac{dR}{dt} = \frac{D}{R} \; \frac{C_\infty^\alpha-\hat{C}^\alpha}{\hat{C}^\beta-\hat{C}^\alpha}
\end{equation}

In the \textit{ideal solution} approximation the concentration at the interface is given by
\begin{equation}
\hat{C}^{\alpha,\beta} = C_e^{\alpha,\beta} \cdot \exp \left\lbrace \frac{R_0}{R} \frac{1-C_e^{\alpha,\beta}}{C_e^\beta - C_e^\alpha}\right\rbrace 
\end{equation}

When nucleation \& growth occur simultaneously the following term has to be added to $dR/dt$ to account for the arrival of new particles of radius $R^*$
\begin{equation}
\frac{dR}{dt} = -\frac{1}{N} \frac{dN}{dt}\; (\delta \cdot R^* - R)
\end{equation}
The factor $\delta\sim 1.05$ accounts for the fact that the radius of new nuclei is slightly higher than $R^*$. In numerical solutions the initial value of $R$ should be $R(t=0) = \delta \cdot R^*$ .

\section{Coarsening}

In the coarsening region the average radius grows as
\begin{equation}
R^3(t) = K\cdot t
\end{equation}
where $K$ is given in the \textit{ideal solution} approximation by
\begin{equation}
K\approx K_{\text{IS}} = \frac{4D R_0}{9}\;\frac{C_e^\alpha\, (1-C_e^\alpha)}{(C_e^\beta - C_e^\alpha )^2}
\end{equation}
Thus the time differential of $R$ is given by
\begin{equation}
\frac{dR}{dt} = \frac{4}{27}\;\frac{D R_0}{R^2}\;\frac{C_e^\alpha\, (1-C_e^\alpha)}{(C_e^\beta - C_e^\alpha )^2}
\end{equation}

\bibliography{pcpsim}

\end{document}